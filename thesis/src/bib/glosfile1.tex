%%%% Just for sake of compatibility this was useful at some points.
%%%% Changes in style needed to make this useful again!

%% For dual entries:
%%	+ Use \gls{<TERM>}
%%	+ Use \glsdesc*{gls-<TERM>} for inline definition
%%  or \begin{Def}\glsdesc*{gls-<TERM>}\end{Def} for explicit definition (leading Def.:)
%%
%% Beginning of sentence \acrlong{<TERM>}
%%	+ Important dualentries can not have \gls in them, themselves!!! BREAKs
%%	+ \glslink{<TERM>}{<OTHER TEXT>} for links!

%% For Glossary use:
%%	+ Use \gls{<TERM>}
%%	+ Use \glsdesc*{<TERM>} for inline definition
%%
%% Important: This breaks if definitions are used inside! Can not have \gls in them, themselves!!! BREAKs
%% or \begin{Def}\glsdesc*{<TERM>}\end{Def} for explicit definition (leading Def.:)
%% Beginning of sentence \acrlong{<TERM>}
%%
%% Acronym specific:
%%	+ \acrfull{<label>}		For long and (acronym)
%%	+ \acrlong{<label>}		For long
%%	+ \acrshort{<label>}	For acronym only

% -------------------------------------------------------------
% Organization, Institue, Company, Group, Place, etc
% -------------------------------------------------------------
\newacronym{LMU}{LMU}{Ludwig-Maximilians Universität München}
\newacronym{TUM}{TUM}{Technische Universität München}
\newacronym{LRZ}{LRZ}{Leibniz Supercomputing Center (Ger. Leibniz-Rechenzentrum)}
\newacronym{IBM}{IBM}{International Business Machines Corporation}
\newacronym{LLNL}{LLNL}{Lawrence Livermore National Laboratory}

% -------------------------------------------------------------
% Research Fields, Topics
% -------------------------------------------------------------
\newacronym{ML}{ML}{Machine Learning}
\newacronym{AI}{AI}{Artificial Intelligence}

% -------------------------------------------------------------
% Related Computer/System Hardwares
% -------------------------------------------------------------
\newacronym{CMG}{CMG}{Core Memory Group}
\newacronym{MMU}{MMU}{Memory Management Unit}
\newacronym{FPGA}{FPGA}{Field Programmable Gate Array}
\newacronym{SVE}{SVE}{Scalable Vector Extension}
\newacronym{HBM2}{HBM2}{High Bandwith Memory 2}
\newacronym{SLES}{SLES}{SUSE Linux Enterprise Server}
\newacronym{IF}{IF}{Interface}
\newacronym{PC}{PC}{Personal Computer}
\newacronym{ESP}{ESP}{Electricity Service Provider}
\newacronym{TDP}{TDP}{Thermal Design Power}
\newacronym{CPU}{CPU}{Central Processing Unit}
\newacronym{GPU}{GPU}{Graphics Processing Unit}
\newacronym{LLC}{LLC}{Last Level Cache}
\newacronym{OSI}{OSI}{Open Systems Interconnection}
\newacronym{ISO/OSI}{ISO/OSI}{International Standard ISO/IEC 7498 Open Systems Interconnection}
\newacronym{ISO}{ISO}{International Organization for Standardization}
\newacronym{RDMA}{RDMA}{Remote Direct Memory Access}
\newacronym{UMA}{UMA}{Uniform Memory Access}
\newacronym{NUMA}{NUMA}{Non-Uniform Memory Access}
%\newacronym{MOSFET}{MOSFET}{Metal-oxide-semiconductor field-effect transistor}
%\newacronym{ALU}{ALU}{Arithmetic Logic Unit}
%\newacronym{NIC}{NIC}{Network Interface Controller}
%\newacronym{DRAM}{DRAM}{Dynamic Random-Access Memory}
%\newacronym{DDR4}{DDR4 SDRAM}{Double Data Rate 4 Synchronous Dynamic Random-Access Memory}
%\newacronym{DIMM}{DIMM}{dual in-line memory module}
%\newacronym{PCI}{PCI}{Peripheral Component Interconnect}
%\newacronym{CSR}{CSR}{Control and Status Register}

% -------------------------------------------------------------
% Related Softwares, Libraries, Programming Models
% -------------------------------------------------------------
\newacronym{HPL}{HPL}{High-Performance Linpack}
\newacronym{RT}{RT}{Runtime}
\newacronym{RM}{RM}{Resource Manager}
\newacronym{EAS}{EAS}{Energy Aware Scheduler}
\newacronym{SLURM}{SLURM}{Simple Linux Utility for Resource Management}
\newacronym{ER}{ER}{Entity-Relationship}
\newacronym{ROM}{ROM}{Read-Only Memory}
\newacronym{APEX}{APEX}{Autonomic Performance Environment for eXascale}
\newacronym{OpenMP}{OpenMP}{Open Multi-Processing}
\newacronym{PGAS}{PGAS}{Partitioned Global Address Space}
\newacronym{OS}{OS}{Operating System}
\newacronym{LDMS}{LDMS}{Lightweight Distributed Metric Service}
\newacronym{DCDB}{DCDB}{Data Center Data Base}
\newacronym{OMNI}{OMNI}{Operations Monitoring and Notification Infrastructure}
\newacronym{ODA}{ODA}{Operational Data Analysis}
\newacronym{KPI}{KPI}{Key Performance Indicator}
\newacronym{VIHPS}{VIHPS}{Virtual Institute -- High Productivity Computing}
\newacronym{MKL}{Intel~MKL}{Intel Math Kernel Library}
\newacronym{MSR}{MSR}{Model Specific Register}
\newacronym{icc}{icc}{Intel C++ Compiler}
\newacronym{RAPL}{RAPL}{Intel's Running Average Power Limit}
\newacronym{BMC}{BMC}{Baseboard Management Controller}
\newacronym{ACPI}{ACPI}{Advanced Configuration and Power Interface}
\newacronym{API}{API}{Application Programming Interface}
\newacronym{OSPM}{OSPM}{Operating System-directed configuration and Power Management}
\newacronym{DVFS}{DVFS}{Dynamic Voltage and Frequency Scaling}
\newacronym{DFS}{DFS}{Dynamic Frequency Scaling}
\newacronym{DVS}{DVS}{Dynamic Voltage Scaling}
\newacronym{BIOS}{BIOS}{Basic Input Output System}
\newacronym{PP}{PP}{Power Plane}
\newacronym{UID}{UID}{User Identifier}
\newacronym{PELT}{PELT}{Per-Entity Load Tracking}
\newacronym{FIFO}{FIFO}{First-In, First-Out}
\newacronym{RAS}{RAS}{Reliability Availability Serviceability}
\newacronym{OASIS-SOA}{OASIS~SOA}{OASIS Reference Model for Service Oriented Architecture}
\newacronym{OASIS}{OASIS}{Organization for the Advancement of Structured Information Standards}
\newacronym{CLI}{CLI}{Command Line Interface}
\newacronym{xCat}{xCat}{Extreme Cloud Administration Toolkit}
\newacronym{DAG}{DAG}{directed acyclic graph}
\newacronym{FSM}{FSM}{finite state machine}
\newacronym{ILP}{ILP}{Instruction-Level Parallelism}
\newacronym{DLP}{DLP}{Data-Level Parallelism}
\newacronym{GPGPU}{GPGPU}{General Purpose Graphics Processing Unit}
\newacronym{SIMD}{SIMD}{single-instruction stream -- multiple-data stream}
\newacronym{SSE}{SSE}{Streaming SIMD Extensions}
\newacronym{PDE}{PDE}{Partial Differential Equation}
\newacronym{NVML}{NVML}{NVIDIA Management Library}
\newacronym{POSIX}{POSIX}{Portable Operating system Interface}
\newacronym{JSON}{JSON}{JavaScript Object Notation}
\newacronym{PTF}{PTF}{Periscope Tuning Framework}
\newacronym{TAU}{TAU}{Tuning and Analysis Utilities}
\newacronym{UML}{UML}{Unified Modelling Language}
\newacronym{XML}{XML}{eXtensible Markup Language}
\newacronym{RCS}{RCS}{real-time control system}

% -------------------------------------------------------------
% Performance Metrics
% -------------------------------------------------------------
\newacronym{GBS}{GBS}{GigaByte per Second}
\newacronym{RPM}{rpm}{revolutions per minute}
\newacronym{QoS}{QoS}{Quality of Service}
\newacronym{IPC}{IPC}{Instructions Per Cycle}
\newacronym{IPS}{IPS}{Instructions Per Second}
\newacronym{OPS}{OPS}{Operations Per Second}

% -------------------------------------------------------------
% Others
% -------------------------------------------------------------
\newacronym{iaw}{i.a.w.}{in accordance with}
\newacronym{EARPLUG}{EARPLUG}{EAR SLURM Plugin}
\newacronym{EARGM}{EARGM}{EAR Global Manager}
\newacronym{EARDB}{EARDB}{EAR Database}

% ---------------------------------------------------
% Dual Entries:
% ---------------------------------------------------
\newdualentry{DLB}
{DLB}{Dynamic Load Balancing}{
	DLB can be understood as a problem or a solution. The problem is defined by parallel execution on multiple processors/nodes, where their load might be unbalanced at runtime. DLB solutions also refer to the approaches to balancing the load.
}

\newdualentry{HPC}
{HPC}{High Performance Computing}{
	High-Performance Computing (HPC) refers to using the computing power of supercomputers or clusters to solve advanced computation problems. Additionally, the current trend also towards the challenges related to big data processing. The main target is to make the challenges of science and engineering problems computable in a reasonable time.
}

\newglossaryentry{HPC system}{
	name={High Performance Computing system},
  text={HPC system},
  description={The computer systems built for \gls{HPC} are called HPC systems. Depending on the design and computing architectures, HPC systems can be distinguished by heterogenous or monogenous architectures and network infrastructures.}
}

\newdualentry{BSP}
{BSP}{Bulk Synchronous Parallel}{
	Bulk synchronous parallel (BSP) is considered as a bridging model for designing parallel algorithms.
}

\newdualentry{MPI+X}
{MPI+X}{MPI+X}{
	It is defined as a hybrid programming model between MPI process + thread. Or we have known that X can be OpenMP threads, Pthread, or other kind of threads.
}

\newdualentry{AMR}
{AMR}{Adaptive Mesh Refinement}{
	In numerical simulation or analysis, Adaptive Mesh Refinement is a method of adapting the accuracy of a solution dynamically. When solutions are calculated numerically, the computation domain are often limited to pre-determined quantified grids or meshes.
}

\newdualentry{MD}
{MD}{Molecular Dynamics}{
	The term indicates molecular dynamic simulations in HPC. It is considered as a computer simulation method for analyzing the physical movements of atoms and molecules.
}

\newdualentry{CRediT}
{CRediT}{Contributor Roles Taxonomy}{
	CRediT (Contributor Roles Taxonomy) is a high-level taxonomy, including 14 roles, that can be used to represent the roles typically played by contributors to research outputs (\url{https://credit.niso.org/}).
}

%\newdualentry{hdf5}
%{HDF5}{Hierarchical Data Format 5}{
%        HDF5 is a standardized set of data formats for large scale complex data
%        objects.
%        HDF5 has associated data model, file format and API as well as a set of
%        tools and APIs and efficient implementations for massive parallel
%        systems.
%}

%\newdualentry{CRAC}
%{CRAC}{Computer Room Air Conditioner}{
%    Computer Room Air Conditioner (CRAC) are
%    air conditioning system using direct expansion refrigeration cycle to
%    actively cool and exchange hot for cold air.
%}
%
%\newdualentry{CRAH}
%{CRAH}{Computer Room Air Handler}{
%    Computer Room Air Handler (CRAH) are
%    air handling systems, used for heat exchange (from water chillers, or ambient air).%
%}

%\newdualentry{PUE}
%{PUE}{Power Usage Efficiency}{%
%    Power Usage Efficiency (PUE) is a ratio defined by the power consumed in the total IT-System divide by
%    the total System Power consumption.
%    $\textit{PUE}=\frac{\text{Total Facility Energy}}{\text{IT Equipment Energy}}$
%}
%\newdualentry{UPS}
%{UPS}{Uninterupted Power Supply}{%
%    A Uninterupted Power Supply (UPS) is a setup to guarantee continous power
%    supply, in the case of a prolonged power outage at the \gls{ESP}.
%    The capacity of the system is set to allow for startup of back-up power generators.
%    For details, refere to.%
%}
%\newdualentry{PSU}
%{PSU}{Power Supply Unit}{%
%    The Power Supply Unit (PSU) is the node's power supply. The PSU provides functionality
%    such as supply voltage conversion (\eg $\SI{230}{\volt}$ to $\SI{12}{\volt}$),
%    as used on node and further converted by \glspl*{VR}.
%    For details, see.%
%}
%\newdualentry{PDU}
%{PDU}{Power Distribution Unit}{
%    Power Distribution Unit (PDU), for power distribution within a rack
%    (which are themselves fed from server room distribution boards).%
%}
%\newdualentry{VR}
%{VR}{Voltage Regulator}{%
%    Voltage Regulators (VRs) provide supply voltage from a node's \gls{PSU} to the \gls{CPU}.
%}
%\newdualentry{PMU}
%{PMU}{Power Management Unit}{%
%    The Power Management Unit (PMU) (as used within this work) is responsible
%    for power management of system, devices, and processor power management
%    (as of ACPI specification).
%    The exact location on the hardware platform, of the associated controller
%    and firmware, differs by implementation.
%    On Intel platforms, some of this functionality is moved inside the CPU
%    and is referred to as Power Control Unit (PCU).
%}
\newdualentry{I/O}
{I/O}{Input/Output}{
    I/O in this work refers to any information transfer from the main
    computational devices, CPU and Memory, to a remote system, such as
    other nodes of other clusters or persistent memory (storage) located outside
    the node.
}

\newdualentry{MPI}
{MPI}{Message Passing Interface}{
	MPI is the standard interface called message passing interface for process communication in HPC systems. Data is moved from the address space of one process to another process through cooperative operations. This standard has gone through a number of revisions and the most recent version is MPI-4.x.
}

%\newdualentry{KISS}
%{KISS}{Keep It Simple Stupid}{%
%    "Keep it simple stupid", attributed to Clarence Leonard (Kelly) Johnson, date
%    unspecified.
%    In popular use by 1970 as the KISS principle.%
%}

\newdualentry{FLOPS}
{FLOPS}{FLoating point Operations Per Second}{%
	Floating point operations per Second (FLOPS) is the metric used for issued
	64 bit floating-point operations per second.
	Precision used is 64 bit in accordance with IEEE754.
	SI-prefixes do apply.%
}
%\newdualentry{PAPI}
%{PAPI}{Performance Application Programming Interface}{%
%	Performance Application Programming Interface (PAPI) provides access to a
%    unified interface to performance counters across heterogeneous
%    architectures.
%    Component PAPI or PAPI-C gives simultaneous access to multiple components'
%    performance data measurements, via a common software interface.%
%}
%\newdualentry{Score-P}
%{Score-P}{Scalable Performance Measurement Infrastructure for Parallel Codes}{%
%	The Score-P project provides the Score-P measurement infrastructure and is a
%    highly scalable and easy-to-use tool suite for profiling, event tracing, and
%    online analysis of HPC applications.%
%}
%\newdualentry{Score-E}
%{Score-E}{Scalable Tools for Energy Analysis and Tuning in HPC}{%
%	The Score-E project extends the Score-P project regarding energy related
%    aspects of analysis and optimization of HPC applications.
%}
%\newdualentry{POMDP}
%{POMDP}{Partially Observable Markov Decision Process}{%
%    A  Partially Observable Markov Decision Process \linebreak (POMDP) is a markov decision process
%    of the form:
%    There exists a finite set of states $S$, a finite set of actions $A$, and a
%    set of observations $Z$, occurring periodically.
%    The system is reward based, where in time period $z \in Z$,
%    an agent in state $s \in S$ chooses an
%    action $a \in A$ and receives a reward based on a reward function $r(s,a)$,
%    while at the same time making a transition to state $s' \in S$ with probability
%    $Pr(s'|s,a) \in [0,1]$.
%    In the following time period $z' \in Z$ an observation is made with probability
%    $Pr(z'|s',a) \in [0,1]$.
%    From these state observations a believe state is derived and an overall system
%    state can be calculated.
%    Fundamentals on mechanisms for solving the general concepts are described
%    in.%
%}
%\newdualentry{DGEMM}
%{DGEMM}{Double-precision GEneral Matrix-Matrix multiplication}{%
%	DGEMM is a general matrix-matrix operation in the form of
%    $C\Leftarrow \alpha A*B+\beta C$.
%	In general vendors provide hardware specific, optimized implementations.%
%}
%
%\newdualentry{OIEP}
%{OIEP}{Open Integrated Energy and Power}{%
%    Open Integrated Energy and Power~--~OIEP
%    refers to a common open model to represent holistic energy
%    and power management systems of HPC systems in an integrated fashion.
%    The concepts of OIEP are used to define a reference model for the
%    description of energy and power management system of HPC systems, the
%    \textit{OIEP reference model}.
%    The OIEP reference model provides common vocabulary and methods to
%    describe management system architectures of planned or existing
%    HPC systems.
%   management system architectures described using the OIEP reference
%    model are called \textit{OIEP architectures}.%
%}
%
%\newdualentry{RFP}
%{RFP}{Request For Proposal}{%
%    A Request For Proposal (RFP), is a document describing requirements for a system
%    procurement at supercomputing facilities.
%    This technical document sets the general information regarding the upcoming
%    procurement, such as dates and deadlines, but also requirements on how the
%    proposed fulfillment of the procurements by the bidder are presented.
%    The main part are the technical requirements for the system to be procured.
%    These requirements are typically weighted in categories of `mandatory',
%    `important' and `target'.
%    An ask for risk assessment of the presented solution can also be part of the
%    RFP.
%    Additional supplement documents may include technical specifications of
%    benchmarks, infrastructure and floor plans.
%    Based on these documents, vendors are asked to make a system proposal with
%    performance and cost estimates for their project bid.
%    The RFP document is sometimes also referred to as `Description of Goods and
%    Services'.%
%}
%\newdualentry{RFI}
%{RFI}{Request For Information}{%
%    For \gls{HPC system} procurement Requests for Information (RFIs) serve to get
%    information about capabilities from vendors.
%    This allows to assess technology capabilities anticipated within a specific
%    time-frame and evaluate how these fit the needs of a center.
%    RFIs help to have an informed perspective regarding asks in \glspl{RFP}.%
%}
%\newdualentry{CPI}
%{CPI}{Cycles Per Instruction}{%
%    Cycles Per Instruction (CPI)  is a measure of how many cycles a single
%    instruction requires to completion.
%	An average value for this metric is computed by reading a reference counter
%    for clock cycles and dividing it by the number of instructions retired over
%    the same time period.%
%}
\newdualentry{HPX}
{HPX}{High Performance ParalleX}{
    The HPX runtime system is the runtime system for the parallel execution
    model ParallelX.%
}
%\newdualentry{CMU}
%{CMU}{CPU Memory Unit}{%
%    A \Gls{CPU} Unit (CMU) are two nodes mounted on a single board.
%    The CMU forms an organizational unit in terms of cooling and power supply of
%    the Fugaku system.
%    (Equivalent to a blade in other systems).%
%}
%\newdualentry{BoB}
%{BoB}{Bunch of Blades}{%
%    A Bunch of Blades (BoB) are 16~\glspl{CMU} is the organization structure of
%    the Fugaku system.%
%}
%\newdualentry{EXA}
%{EXA}{EXecution unit A}{%
%    Execution unit A (EXA), is one of two integer execution units, present in each
%    cores.
%    These are referred to as \texttt{EXA} and  \texttt{EXB}.
%}
%\newdualentry{FLA}
%{FLA}{FLoatinting-point unit A}{
%    Floating-point unit A (FLA), is one of two floating-point units, present in each cores.
%    These are referred to as \texttt{FLA} and \texttt{FLB}.
%}
%\newdualentry{IIL}
%{IIL}{Instruction Issuance Limit}{
%    Instruction Issuance Limit (IIL) is a
%    mode of operation where the instruction issue width of the processor (per core) is limited to from four to two.
%}

