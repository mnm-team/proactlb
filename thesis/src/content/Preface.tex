\chapter{Preface}
\label{ch:Preface}

% Contemporary society is relying on the ``power of computation'' without giving it a second thought, even though computing machinery has only very recently entered the toolset of mankind:
%Babbage's analytical engine is often regarded as the first computing machine,
%which he conceived in 1833~\cite[42]{Campbell-Kelly2013}.
%In 1941 Zuse built the first binary based programmable Turing complete
%computer, the Z3~\cite[55]{Zuse2010}.

%Today, computer \glspl{chip} in smartphones, laptop computers and embedded
%systems make it ubiquitous to generate, send and receive data and process
%information anywhere.
%Where local processing power, its performance or storage capability, is not
%sufficient, distributed resources are used.
%
%The seemingly simple task of getting an accurate weather forecast on a mobile
%phone is only possible by querying information from large simulations.
%These simulations are not able to run on one's phone but require dedicated
%computer systems.
%By operating \gls{HPC} \glslink{HPC system}{systems}, which run these
%simulations in regular intervals (\eg several times a day) current and accurate
%forecasts are possible.
%The quality of \gls{NWP} is in direct correlation with the performance of the
%computational systems available to the national meteorological and hydrological
%services.
%In 1950 the first weather models were calculated, however only by the 1970s it
%was feasible to solve the full set of equations for weather forecasting
%as proposed by Abbe~\cite{Abbe1901} and Bjerknes~\cite{Bjerknes1904} as early as
%1901.
%Since then, every decade added one day of useful forecast by improvements to
%both the numerical models and the performance of the computer
%systems~\cite{Bauer2015}.
%This public commodity is driven by the progress in computation:
%\begin{quote}
%    ``Computation is essential in everything
%    discussed here[\footnote{%
%        The authors in \cite{Alley2019} specifically talk about numerical
%        modeling in weather prediction.
%        However, this can likely be generalized to any scientific discipline
%        which has to rely on theory, experiment and modeling \& simulation.}].
%    Progress will involve larger
%    ensembles of model runs at higher resolution
%    leading to improved probabilistic forecasts,
%    including those of hazardous weather. This
%    can be realized if governments maintain a
%    steady schedule of investment in high-speed
%    computing, recognizing the strong evidence
%    that such investments will be repaid many
%    times over in savings to the economy.''
%    \newline\strut\hfill Alley, Emanuel and Zhang, 2019~\cite[344]{Alley2019}
%\end{quote}
%The impact to society justifies the investment in large computing resources:
%Recent estimates of the cost to benefit ratio for national weather
%services range from $1:3$ to $1:10$, above all by avoiding weather related
%damages and guiding decision making~\cite{Perrels2013}.
%To reach the performance required for \gls*{NWP} the weather services employ
%computer systems which are among the fastest \glspl*{HPC system} in the
%world~\cite{Bauer2015}.
%In their review on ``the quiet revolution of numerical weather prediction'',
%Bauer \etal~\cite{Bauer2015} discuss the scientific challenges for reaching and
%keeping the pace of improvements for \gls*{NWP}.
%Ensemble models and an increased simulation granularity in the order of $100$ to
%$1000$ regarding computational tasks are likely for the next 10 years.
%
%Regarding technological challenges, co-relating this development with processor
%performance improvements, historically going hand in hand with the advances of
%Moore's law~\cite{Moore1965}, the energy consumption will
%increase accordingly:
%To be able to simulate the anticipated weather models with todays technology
%will require 10~times more power.
%In the review by Bauer \etal the researchers mention an upper limit for
%affordable power of centers such as the \gls{ECMWF} of about
%$\si{\num{20}\mega\volt\ampere}$~\cite{Bauer2015} or $\si{\num{20}\mega\watt}$.
%
%In other words, the colloquial ``power of computation'' -- performance, as
%stated above -- is tightly coupled to actual power, the consumption of energy.
%This is critical for computing centers reaching the scales of industrial energy
%consumers, capable of generating high dynamic loads~\cite{Stewart2019}.
%
%\Pne is among the primary issues to resolve for the \gls{exascale era}, the era
%of computers reaching \num{e18}~\gls{FLOPS}.
%
%In this work the author strives to contribute a small aspect to computer science
%and engineering to move beyond a theoretical hurdle on the path to extend what
%is possible with \glspl*{HPC system}.

Lorem ipsum dolor sit amet, consectetur adipiscing elit. Sed congue vulputate lobortis. Praesent justo eros, gravida eget enim eu, rhoncus maximus nisi. Suspendisse sed ipsum et arcu sodales accumsan eget id elit. Integer mattis, eros nec maximus aliquam, urna lacus suscipit purus, nec cursus erat libero quis nibh. Ut et vehicula tellus. Praesent laoreet accumsan velit, nec tincidunt neque consectetur sit amet. Phasellus sodales nulla leo, vitae sodales lectus efficitur vel. Aenean consequat tortor sed enim congue dapibus.

Ut dolor neque, imperdiet eget eleifend a, pretium non risus. Interdum et malesuada fames ac ante ipsum primis in faucibus. Mauris consequat efficitur libero. Morbi ac efficitur neque. Nulla ac nulla ut dui aliquet condimentum vel et urna. Proin vehicula lobortis auctor. Mauris nibh felis, gravida at commodo ut, lacinia et libero. Vestibulum congue, tellus quis lobortis pharetra, nibh nisi placerat urna, a consectetur massa sem non quam.

Integer placerat dictum orci, nec tincidunt dui. Praesent ut massa ligula. Proin iaculis in ipsum et iaculis. Integer eu diam justo. Duis id tincidunt ligula. Aliquam tempus, tellus ornare malesuada tempor, libero justo cursus dui, ac convallis neque sapien et leo. Nunc et mauris at purus bibendum rutrum non vitae lectus. Integer ornare non sapien sit amet interdum. Integer egestas dolor tempus risus venenatis blandit. Etiam semper gravida orci eu gravida. Cras vitae odio eu ipsum fermentum pharetra. Donec a pulvinar arcu, et dignissim sem. Praesent non porta dui. Curabitur congue orci vitae consequat auctor. Mauris vel posuere elit.

Donec vitae diam magna. Nunc ullamcorper, massa eu consequat feugiat, eros velit bibendum mauris, et ullamcorper libero diam id nisl. Aliquam facilisis interdum sagittis. Praesent blandit felis vel aliquet varius. Nullam a commodo libero. Praesent commodo dapibus ultrices. Vestibulum in dolor massa.

Vivamus non diam eleifend, dictum est sit amet, ullamcorper orci. Duis vitae feugiat felis. Aenean elementum lacinia purus, id gravida metus. Ut vehicula est justo, id lobortis turpis dapibus vitae. Nunc mollis accumsan magna ut luctus. Duis ac suscipit lacus, non elementum massa. Ut tempor magna nec faucibus ullamcorper. Nunc sodales facilisis malesuada.
