\section{Publications}
\label{sec:intro_publication}
\index{Intro!Publication}

The following publications are associated with the thesis. We split the publications into two groups linked directly or indirectly to the thesis. We refer to the contributor roles taxonomy named \textit{CRediT} at \url{https://casrai.org/credit/} to show the author's contribution to each paper.\\

There are some main roles used to describe the contribution:
\begin{itemize}
	\item \textit{\textbf{Conceptualization}}: indicates the contribution of ideas, formulation, or evolution of overarching research goals.
	\item \textit{\textbf{Data curation}}: points to the management activities around data, e.g., annotate data, scrub data, maintain research data, etc.
	\item \textit{\textbf{Formal analysis}}: means using statistical, mathematical, computational, or other formal techniques to analyze the data.
	\item \textit{\textbf{Investigation}}: is to conduct a research and investigation process, explicitly performing the experiments or data collection.
	\item \textit{\textbf{Methodology}}: is the contribution of creating or designing methodology, model.
	\item \textit{\textbf{Validation}}: indicates the verification as a part of the activity or the overall reproducibility of results/experiments.
	\item \textit{\textbf{Visualization}}: is the preparation, creation, and presentation of the published work, e.g., visualization, data presentation.
	\item \textit{\textbf{Writing} – \textbf{original draft}}: is specifically to write the initial draft (including substantive translation). This is also the contribution of creating and preparing the published work.
	\item \textit{\textbf{Writing} – \textbf{review \& editing}}: contribute to the published work by critical review, commentary, or revision (including pre- or post-publication stages).
\end{itemize}

\textbf{(A) Publications directly associated with the dissertation:}

\textit{``From Reactive to Proactive Load Balancing for Task-based Parallel Applications in Distributed Memory Machines.'' Journal of Concurrency and Computation: Practice and Experience (CCPE), 2023,  \url{https://doi.org/10.1002/cpe.7828}.}
\begin{itemize}
	\item \textit{Authors}: Minh Thanh Chung, Josef Weidendorfer, Karl Fürlinger, and Dieter Kranzlmüller
	\item \textit{Paper's contribution/Abstract}: First, this paper proposes a performance model to analyze reactive balancing behaviors and understand the bound leading to incorrect decisions. Second, we introduce a proactive approach to further improve balancing tasks at runtime. The approach also exploits task-based programming models with a dedicated thread (named $Tcomm$). Nevertheless, the main idea is to force $Tcomm$ not only to monitor load; it will characterize tasks and train load prediction models by online learning. ``Proactive'' indicates offloading tasks with an approriate number at once to a potential victim (denoted by an underloaded/fast process). The experimental results confirm speedup improvements from $1.5\times$ to $3.5\times$ in important use cases compared to the previous solutions. Furthermore, this approach can support co-scheduling tasks across multiple applications.
	\item \textit{Relation to thesis}: This paper is directly related to RQ1 and SQ2 in this thesis. We show how to perform task characterization and generate load prediction at runtime. Then, this information is applied to a proactive algorithm which guides task offloading to balance the load.
	\item \textit{Author's contribution}: Conceptualization, Data curation, Formal analysis, Investigation, Methodology, Validation, Visualization, Writing – original draft.
\end{itemize}

\textit{``Proactive Task Offloading for Load Balancing in Iterative Applications.'' The 14th International Conference on Parallel Processing and Applied Mathematics (PPAM22), 2022, \url{https://doi.org/10.1007/978-3-031-30442-2_20}.}
\begin{itemize}
	\item \textit{Authors}: Minh Thanh Chung, Josef Weidendorfer, Karl Fürlinger, and Dieter Kranzlmüller
	\item \textit{Paper's contribution/Abstract}: Load imbalance is often a challenge for applications in parallel systems. Static cost models and pre-partitioning algorithms distribute the load at the beginning. However, during execution, dynamic changes or inaccurate cost indicators may lead to imbalance at runtime. Reactive work-stealing strategies can help monitor the execution and perform task migration to balance the load. The benefits depend on the migration overhead and assumption about future execution. Our proactive approach further improves existing solutions by applying machine learning to online load prediction. We propose a fully distributed algorithm for adapting the prediction result to guide task offloading. The experiments are performed with an artificial test case and a realistic application named Sam(oa)$^2$ on three systems with different communication latencies. The results confirm that improvements can be achieved for important use cases compared to previous methods. Furthermore, this approach can support co-scheduling tasks across multiple applications.
	\item \textit{Relation to thesis}: This paper is directly related to RQ2. We show how to perform online task characterization and generate the load prediction at runtime. Then, this information is applied to a proactive algorithm which guides task offloading to balance the load.
	\item \textit{Author's contribution}: Conceptualization, Data curation, Formal analysis, Investigation, Methodology, Validation, Visualization, Writing – original draft.
\end{itemize}

\textit{``User-defined Tools for Characterizing Task-Parallel Applications and Predicting Load Imbalance.'' In 2021 15th International Conference on Advanced Computing and Applications (ACOMP21), pp.98-105. IEEE, 2021, \url{https://doi.org/10.1109/ACOMP53746.2021.00020}.}
\begin{itemize}
	\item \textit{Authors}: Minh Thanh Chung and Dieter Kranzlmüller
	\item \textit{Paper's contribution/Abstract}: Parallel applications can be built portably on heterogeneous shared or distributed memory systems using task-based programming models. The decomposition into tasks allows to address load imbalance in parallel programs more easily, if knowledge about the application’s characteristics can be obtained. In our paper, we introduce an approach for characterizing tasks at runtime using callback functions and a dedicated thread per rank for tool isolation with minimal perturbation of the target application’s execution. With the characterization, prediction of execution time can be achieved using a machine learning approach. This serves as input for repartitioning or migrating the tasks such that the overall execution time can be improved. The results of our experiments using microbenchmarks and realistic applications confirm the benefits of our solution, in which the predicted information can adapt dynamic load balancing to large-scale use-cases.
	\item \textit{Relation to thesis}: This paper is directly related to SQ1. We propose a practical scheme of how online task characterization and load prediction work. The results confirm that our design is lightweight and feasible to adapt this information to load balancing algorithms.
	\item \textit{Author's contribution}: Conceptualization, Data curation, Formal analysis, Investigation, Methodology, Validation, Visualization, and Writing – original draft.
\end{itemize}

\textit{``Predictive, reactive and replication-based load balancing of tasks in Chameleon and Sam(oa)2.'' In Proceedings of the Platform for Advanced Scientific Computing Conference (PASC21), pp.1-10, 2021, \url{https://doi.org/10.1145/3468267.3470574}.}
\begin{itemize}
	\item \textit{Authors}: Philipp Samfass, Jannis Klinkenberg, Minh Thanh Chung, and Michael Bader.
	\item \textit{Paper's contribution/Abstract}: Increasingly complex hardware architectures as well as numerical algorithms make balancing load in parallel numerical software for adaptive mesh refinement an inherently difficult task, especially if variability of system components and unpredictability of execution time comes into play. Yet, traditional \textit{predictive} load balancing strategies are largely based on cost models that aim to predict the execution time of computational tasks. To address this fundamental weakness, we present a novel \textit{reactive} load balancing approach in distributed memory for MPI+OpenMP parallel applications that is based on keeping tasks speculatively replicated on multiple MPI processes. Replicated tasks are scheduled fully reactively without the need of a predictive cost model. Task cancellation mechanisms help to keep the overhead of replication minimal by avoiding redundant computation of replicated tasks. We implemented our approach in the Chameleon library for reactive load balancing. Our experiments in the parallel dynamic adaptive mesh refinement software sam(oa)$^2$ demonstrate performance improvements in the presence of wrong cost models and artificially introduced noise to simulate imbalances coming from hardware variability.
	\item \textit{Relation to thesis}: The relevance of this paper is indirect in that we figure out the limit of current reactive approaches and open a hypothesis of proactive approach.
	\item \textit{Author's contribution}: Data curation, Formal analysis, Investigation, Validation, Visualization, and Writing - review \& editing
\end{itemize}

\textit{``Scheduling across multiple applications using task-based programming models.'' In 2020 IEEE/ACM Fourth Annual Workshop on Emerging Parallel and Distributed Runtime Systems and Middleware (IPDRM), IEEE, pp.1-8. In conjunction with the International Conference for High Performance Computing, Networking, Storage and Analysis (SC20), November 2020, \url{https://doi.org/10.1109/IPDRM51949.2020.00005}.}
\begin{itemize}
	\item \textit{Authors}: Minh Thanh Chung, Josef Weidendorfer, Philipp Samfass, Karl Fürlinger and Dieter Kranzlmüller
	\item \textit{Paper's contribution/Abstract}: Task-based programming models have shown their potential for efficiency and scalability in parallel and distributed systems. With such a model, a parallel application is broken down into a graph of tasks, which are subsequently scheduled for execution. Recently, implementations of task-based models have addressed distributed memory and heterogeneous systems with accelerators. However, the problem of scheduling tasks as well as allocating resources at runtime is still a challenge. In this paper, we propose coordinated and cooperative task scheduling across multiple applications. The main idea is to exploit the application's idle time e.g. from imbalance to serve tasks from another application. The experiments use Chameleon, a task-based framework for reactive tasking in distributed memory systems. In various example scenarios, we show improvements in CPU utilization of 5\%-15\% by coordinated scheduling.
	\item \textit{Relation to thesis}: This paper is directly related to SQ2 in this thesis. We show an idea and a methodology of co-scheduling tasks across multiple applications instead of balancing only the load in a single application. The paper reveals our analysis and how the scheme works in practice. For the evaluation, our results confirm a benefit of 5\%-15\% improvement compared to the baseline.
	\item \textit{Author's contribution}: Conceptualization, Data curation, Formal analysis, Investigation, Methodology, Validation, Visualization, and Writing - original draft.
\end{itemize}

%\textbf{(B) Standardization Effort during this dissertation:}
%
%\textit{Nikola Nincic, Evaluation of Modern PGAS Libraries for Work Stealing in Distributed Memory (Bachelor Supervision), co-supervise with Philipp Samfass (TUM), July 2021.}
%\begin{itemize}
%	\item \textit{Abstract}: The work focuses on performance evaluation between different distributed memory data structures. The experiments are performed on three HPC systems named CoolMUC2, SuperMUC-NG, and BEAST-system at the Leibniz Supercomputing Centre (LRZ).
%	\item \textit{Summary}: As mentioned, communication latency is one constraint of the problem in the thesis. However, the built-in technologies (e.g., RDMA InfiniBand) can help, and we want to investigate the possibilities around the dynamic balancing problem.
%\end{itemize}

\textbf{(B) Outside the scope of this dissertation:}

\textit{``A Profiling-based Approach to Cache Partitioning of Program Data.'' In the 23rd International Conference on Parallel and Distributed
Computing, Applications and Technologies (PDCAT’22), 2022, \url{https://doi.org/10.1007/978-3-031-29927-8_35}.}
\begin{itemize}
	\item \textit{Authors}: Sergej Breiter, Josef Weidendorfer, Minh Thanh Chung, and Karl Fürlinger 
	\item \textit{Abstract}: Cache efficiency is important to avoid unnecessary data transfers and to keep processors active. Cache partitioning, a technique to virtually divide a cache into multiple partitions, has become available in recent hardware. Cache partitioning can improve efficiency by isolating data with high temporal locality to avoid its early eviction before reuse. However, deciding on the partitioning is challenging, because it depends on the locality of reference. To facilitate the decision-making, we propose a profiling-based approach that measures locality, providing knowledge for cache partitioning without requiring manual code analysis. We present a profiling tool and confirm its benefits through experiments on Fujitsu's A64FX processor, which supports the cache partitioning mechanism called \textit{sector cache}. Our results show ways to optimize program codes to improve cache efficiency.
	\item \textit{Author's contribution}: Formal analysis, Validation, Visualization, and Writing - review \& editing.
\end{itemize}

\textit{``From Transcripts to Insights for Recommending the Curriculum to University Students.'' SN Computer Science Journal (1), No.6, pp.1-14, 2020, \url{https://doi.org/10.1007/s42979-020-00332-7}.}
\begin{itemize}
	\item \textit{Authors}: Thong Le Mai, Minh Thanh Chung, Van Thanh Le, and Nam Thoai
	\item \textit{Abstract}: Student data plays an important role in evaluating the effectiveness of educational programs in the universities. All data is aggregated to calculate the education criteria by year, region, or organization. Remarkably, recent researches showed the data impacts when making exploration to predict student performance objectives. Many methods in terms of data mining were proposed to be suitable to extract useful information in regards to data characteristics. However, the reconciliation between applied methods and data characteristics still exists some challenges. Our paper will demonstrate the analysis of this relationship for a specific dataset in practice. The paper describes a distributed framework based on Spark for extracting information from raw data. Then, we integrate machine learning techniques to train the prediction model. The experiments results are analyzed through different scenarios to show the harmony between the influencing factors and applied techniques.
	\item \textit{Author's contribution}: Conceptualization, Data curation, Formal analysis, Investigation, Methodology, Validation, Visualization, and Writing - original draft.
\end{itemize}

\newpage
